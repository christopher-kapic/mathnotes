\documentclass{article}
\usepackage[utf8]{inputenc}
\usepackage{amsmath}
\usepackage{amssymb}
\usepackage{amsthm}
\usepackage[shortlabels]{enumitem}
\usepackage[margin=1in]{geometry}
\setlist*[enumerate]{label=(\alph*)}
\usepackage{tikz-cd}
\newtheorem{definition}{Definition}
\newtheorem{theorem}{Theorem}
\newtheorem{hint}{Hint}
\newtheorem{lemma}{Lemma}

\title{Math 320 Final Notes - Week 1}
\author{Christopher Kapic}
\date{December 2020}

\begin{document}

\maketitle

\section*{1.2}
\begin{theorem}
    The triangle inequality states that \[|a-b|\leq |a-c|+|c-b|.\]
\end{theorem}

\begin{theorem}
    Two real numbers $a$ and $b$ are equal if and only if for every real number $\epsilon > 0$ it follows that $|a-b|<\epsilon$.
\end{theorem}

\section*{1.3 - The Axiom of Completeness}
\begin{theorem}[Axiom of Completeness]
    Every nonempty set of real numbers that is bounded has a least upper bound.
\end{theorem}

\begin{definition}[bounded]
    A set $A \subseteq \mathbb{R}$ is \textit{bounded above} if there exists a number $b \in \mathbb{R}$ such that $a \leq b$ for all $a \in A$. The number $b$ is called an \textit{upper bound} for $A$.

    Similarly, the set $A$ is \textit{bounded below} if there exists a \textit{lower bound} $l \in \mathbb{R}$ satisfying $l \leq a$ for every $a \in A$.
\end{definition}

\begin{definition}[least upper bound, supremum]
    A real number $s$ is the \textit{least upper bound} for a set $A \subseteq \mathbb{R}$ if it meets the following two criteria:
    \begin{enumerate}[(i)]
        \item $s$ is an upper bound for $A$;
        \item if $b$ is any upper bound for $A$, then $s \leq b$.
    \end{enumerate}
    The least upper bound is also known as the \textit{supremum} of the set $A$.
\end{definition}

\begin{definition}[maximum]
    A real number $a_0$ is a \textit{maximum} of the set $A$ is $a_0$ is an element of $A$ and $a_0\geq a$ for all $a \in A$. Similarly, a number $a_1$ is a \textit{minimum} of $A$ if $a_1\in A$ and $a_1 \leq a$ for every $a \in A$.
\end{definition}

\begin{lemma}
    Assume $s \in \mathbb{R}$ is an upper bound for a set $A \subseteq \mathbb{R}$. Then, $s = \sup A$ if and only if, for every choice of $\epsilon > 0$, there exists an element $a \in A$ satisfying $s-\epsilon < a$.
\end{lemma}

\section*{1.4 - Consequences of Completeness}
\begin{theorem}[Nested Interval Property]
    For each $n \in \mathbb{R}$, assume we are given a closed interval $I_n=[a_n,b_n]=\{x \in \mathbb{R}:a_n \leq x \leq b_n\}$. Assume also that each $I_n$ contains $I_{n+1}$. Then, the resulting nested sequence of closed intervals \[I_1\supseteq I_2 \supseteq I_3 \supseteq I_4 \supseteq \dots\] has a nonempty intersection; that is, $\bigcap _{n=1}^\infty I_n \neq \emptyset$.
\end{theorem}

\begin{theorem}[Archimedean Property]
    \begin{enumerate}[(i)]
        \item Given any number $x \in \mathbb{R}$, there exists an $n \in \mathbb{N}$ satisfying $n > x$.
        \item Given any real number $y > 0$, there exists an $n \in \mathbb{N}$ satisfying $1/n<y$.
    \end{enumerate}
\end{theorem}

\begin{theorem}[Density of $\mathbb{Q}$ in $\mathbb{R}$]
    For every two real numbers $a$ and $b$ with $a < b$, there exists a rational number $r$ satisfying $a < r < b$.

    Furthermore, given any two real numbers $a < b$, there exists an irrational number $t$ satisfying $a < t < b$.
\end{theorem}

\begin{theorem}[$\sqrt{2}$ exists]
    There exists a real number $\alpha \in \mathbb{R}$ satisfying $\alpha ^2 = 2$.
\end{theorem}


\end{document}