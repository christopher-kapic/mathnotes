\documentclass{article}
\usepackage[utf8]{inputenc}
\usepackage{amsmath}
\usepackage{amssymb}
\usepackage{amsthm}
\usepackage[shortlabels]{enumitem}
\usepackage[margin=1in]{geometry}
\setlist*[enumerate]{label=(\alph*)}
\usepackage{tikz-cd}
\newtheorem{definition}{Definition}
\newtheorem{theorem}{Theorem}
\newtheorem{lemma}{Lemma}
\newtheorem{hint}{Hint}

\title{Math 320 Final Notes - Week 6}
\author{Christopher Kapic}
\date{December 2020}

\begin{document}

\maketitle

\section*{3.2 - Open and Closed Sets}
\begin{definition}[open]
    A set $O \subseteq \mathbb{R}$ is \textit{open} if for all points $a \in O$ there exists an $\epsilon$-neighborhood $V_\epsilon (a) \subseteq O$.
\end{definition}

\begin{theorem}[3.2.3]
    \begin{enumerate}[(i)]
        \item The union of an arbitrary collection of open sets is open.
        \item The intersection of a finite collection of open sets is open.
    \end{enumerate}
\end{theorem}

\begin{definition}[limit point]
    A point $x$ is a \textit{limit point} of a set $A$ if every $\epsilon$-neighborhood $V_\epsilon (x)$ of $x$ intersects the set $A$ at some point other than $x$.
\end{definition}

\begin{theorem}[3.2.5]
    A point $x$ is a limit point of a set $A$ if and only if $x = \lim a_n$ for some sequence $(a_n)$ contained in $A$ satisfying $a_n \neq x$ for all $n \in \mathbb{N}$.
\end{theorem}

\begin{definition}[isolated point]
    A point $a \in A$ is an \textit{isolated point} of $A$ if it is not a limit point of $A$.
\end{definition}

\begin{definition}[closed]
    A set $F \subseteq \mathbb{R}$ is \textit{closed} if it contains its limit points.
\end{definition}

\begin{theorem}[3.2.8]
    A set $F \subseteq \mathbb{R}$ is closed if and only if every Cauchy sequence contained in $F$ has a limit that is also an element of $F$.
\end{theorem}

\begin{theorem}[Density of $\mathbb{Q}$ in $\mathbb{R}$]
    For every $y \in \mathbb{R}$, there exists a sequence of rational numbers that converges to $y$.
\end{theorem}

\begin{definition}[closure]
    Given a set $A \subseteq \mathbb{R}$, let $L$ be the set of all limit points of $A$. The \textit{closure} of $A$ is defined to be $\overline{A}=A \cup L$.
\end{definition}


\begin{theorem}[3.2.12]
    For any $A \subseteq \mathbb{R}$, the closure $\overline{A}$ is a closed set and is the smallest closed set containing $A$.
\end{theorem}

\begin{theorem}[3.2.13]
    A set $O$ is open if and only if $O^c$ is closed. Likewise, a set $F$ is closed if and only if $F^c$ is open.
\end{theorem}

\begin{theorem}[3.2.14]
    \begin{enumerate}[(i)]
        \item The union of a finite collection of closed sets is closed.
        \item The intersection of an arbitrary collection of closed sets is closed.
    \end{enumerate}
\end{theorem}

\section*{3.3 - Compact Sets}
\begin{definition}[compactness]
    A set $K \subseteq \mathbb{R}$ is \textit{compact} if every sequence in $K$ has a subsequence that converges to a limit that is also in $K$.
\end{definition}

\begin{definition}[bounded]
    A set $A \subseteq \mathbb{R}$ is \textit{bounded} if there exists $M > 0$ such that $|a| \leq M$ for all $a \in A$.
\end{definition}

\begin{theorem}[Characterization of Compactness in $\mathbb{R}$ - 3.3.4]
    A set $K \subseteq \mathbb{R}$ is compact if and only if it is closed and bounded.

    Proof on page 96.
\end{theorem}

\begin{theorem}[Nested Compact Set Property - 3.3.5]
    If \[K_1 \supseteq K_2 \supseteq K_3 \supseteq K_4 \supseteq \dots \] is a nested sequence of nonempty compact sets, then the intersection $\bigcap _{n=1}^\infty K_n$ is not empty.

    Proof on page 97.
\end{theorem}

\begin{definition}[open cover, finite subcover]
    Let $A \subseteq \mathbb{R}$. An \textit{open cover} for $A$ is a (possibly infinite) collection of open sets $\{O_\lambda : \lambda \in \Lambda \}$ whose union contains the set $A$; that is, $A \subseteq \bigcup _{\lambda \in \Lambda} O_\lambda$. Given an open cover for $A$, a \textit{finite subcover} is a finite subcollection of open sets from the original open cover whose union still manages to completely contain $A$.
\end{definition}

\begin{theorem}[Heine-Borel Theorem - 3.3.8]
    Let $K$ be a subset of $\mathbb{R}$. All of the following statements are equivalent in the sense that any one of them implies the two others:
    \begin{enumerate}[(i)]
        \item $K$ is compact.
        \item $K$ is closed and bounded.
        \item Every open cover for $K$ has a finite subcover.
    \end{enumerate}

    Proof on page 98.
\end{theorem}

\section*{3.4 - Perfect Sets and Connected Sets}
\begin{definition}[perfect]
    A set $P \subseteq \mathbb{R}$ is \textit{perfect} if it is closed and contains no isolated points.
\end{definition}

\begin{theorem}[3.4.3]
    A nonempty perfect set is uncountable.

    Proof on page 102.
\end{theorem}

\begin{definition}[separated, disconnected, connected]
    Two nonempty sets $A,B \subseteq \mathbb{R}$ are \textit{separated} if $\overline{A}\cap B$ and $A \cap \overline{B}$ are both empty. A set $E \subseteq \mathbb{R}$ is \textit{disconnected} if it can be written as $E = A \cup B$, where $A$ and $B$ are nonempty separated sets.

    A set that is not disconnected is called a \textit{connected} set.
\end{definition}

\begin{theorem}[3.4.6]
    A set $E \subseteq \mathbb{R}$ is connected if and only if, for all nonempty disjoint sets $A$ and $B$ satisfying $E=A\cup B$, there always exists a convergent sequence $(x_n)\rightarrow x$ with $(x_n)$ contained in one of $A$ or $B$, and $x$ an element of the other.

    Proof on page 104.
\end{theorem}

\begin{theorem}[3.4.7]
    A set $E \subseteq \mathbb{R}$ is connected if and only if whenever $a < c < b$ with $a,b \in E$, it follows that $c \in E$ as well.

    Proof on page 105.
\end{theorem}

\end{document}