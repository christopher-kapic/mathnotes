\documentclass{article}
\usepackage[utf8]{inputenc}
\usepackage{amsmath}
\usepackage{amssymb}
\usepackage{amsthm}
\usepackage[shortlabels]{enumitem}
\usepackage[margin=1in]{geometry}
\setlist*[enumerate]{label=(\alph*)}
\usepackage{tikz-cd}
\newtheorem{definition}{Definition}
\newtheorem{theorem}{Theorem}
\newtheorem{lemma}{Lemma}
\newtheorem{hint}{Hint}

\title{Math 320 Final Notes - Week 7}
\author{Christopher Kapic}
\date{December 2020}

\begin{document}

\maketitle

\section*{4.2 - Functional Limits}
\begin{definition}[functional limit]
    Let $f:A\rightarrow \mathbb{R}$, and let $c$ be a limit point of the domain $A$. We say that $\lim _{x\rightarrow c}f(x)=L$ provided that, for all $\epsilon > 0$, there exists a $\delta > 0$ such that whenever $0 < |x-c| < \delta$ (and $x \in A$) it follows that $|f(x)-L < \epsilon$.
\end{definition}

\begin{definition}[functional limit: topological version]
    Let $c$ be a limit point of the domain of $f:A\rightarrow \mathbb{R}$. We say $\lim _{x\rightarrow c}f(x)=L$ provided that, for every $\epsilon$-neighborhood $V_\epsilon (L)$ of $L$, there exists a $\delta$-neighborhood $V_\delta (c)$ around $c$ with the property that for all $x \in V_\delta (c)$ different from $c$ (with $x \in A$) it follows that $f(x) \in V_\epsilon (L)$.
\end{definition}

\begin{theorem}[Sequential Criterion for Functional Limits - 4.2.3]
    Given a function $f:A\rightarrow \mathbb{R}$ and a limit point $c$ of $A$, the following two statements are equivalent:
    \begin{enumerate}[(i)]
        \item $\lim _{x\rightarrow c}f(x)=L$.
        \item For all sequences $(x_n) \subseteq A$ satisfying $x_n \neq c$ and $(x_n) \rightarrow c$, it follows that $f(x_n)\rightarrow L$.
    \end{enumerate}

    Proof on page 118.
\end{theorem}

\begin{theorem}[Corollary 4.2.4 - Algebraic Limit Theorem for Functional Limits]
    Let $f$ and $g$ be functions defined on a domain $A \subseteq \mathbb{R}$, and assume $\lim _{x\rightarrow c}f(x)=L$ and $\lim _{x\rightarrow c}g(x)=M$ for some limit point $c$ of $A$. Then,
    \begin{enumerate}[(i)]
        \item $\lim _{x\rightarrow c}kf(x)=kL$ for all $k \in \mathbb{R}$,
        \item $\lim _{x\rightarrow c}[f(x)+g(x)]=L+M$,
        \item $\lim _{x\rightarrow c}[f(x)g(x)]=LM$, and
        \item $\lim _{x\rightarrow c}f(x)/g(x) = L/M$, provided $M \neq 0$.
    \end{enumerate}

    Proof on page 119.
\end{theorem}

\section*{4.3 - Continuous Functions}
\begin{definition}[continuity]
    A function $f:A\rightarrow \mathbb{R}$ is \textit{continuous} at a point $c\in A$ if, for all $\epsilon > 0$, there exists a $\delta > 0$ such that whenever $|x-c|<\delta$ (and $x \in A$) it follows that $|f(x)-f(c)| < \epsilon$.

    If $f$ is continuous at every point in the domain $A$, then we say that $f$ is \textit{continuous} on $A$.
\end{definition}

\begin{theorem}[Characterizationf of Continuity - 4.3.2]
    Let $f:A\rightarrow \mathbb{R}$, and let $c \in A$. The function $f$ is continuous at $c$ if and only if any one of the following three conditions is met:
    \begin{enumerate}[(i)]
        \item For all $\epsilon > 0$, there exists a $\delta > 0$ such that $|x - c| < \delta$ (and $x \in A$) implies $|f(x)-f(c)|<\epsilon$;
        \item For all $V_\epsilon (f(c))$, there exists a $V_\delta (c)$ with the property that $x \in V_\delta (c)$ (and $x \in A$) implies $f(x) \in V_\epsilon (f(c))$;
        \item For all $(x_n)\rightarrow c$ (with $x_n \in A$), it follows that $f(x_n) \rightarrow f(c)$.
        
        If $c$ is a limit point of $A$, then the above conditions are equivalent to\dots
        \item $\lim _{x \rightarrow c}f(x) = f(c)$.
        
        Proof on page 123.
    \end{enumerate}
\end{theorem}

\begin{theorem}[Corollary 4.3.3 - Criterion for Discontinuity]
    Let $f:A\rightarrow \mathbb{R}$, and let $c \in A$ be a limit point of $A$. If there exists a sequence $(x_n)\subseteq A$ where $(x_n)\rightarrow c$ but such that $f(x_n)$ does not converge to $f(c)$, we may conclude that $f$ is not continuous at $c$.
\end{theorem}

\begin{theorem}[Algebraic Continuity Theorem - 4.3.4]
    Assume $f:A\rightarrow \mathbb{R}$ and $g:A\rightarrow \mathbb{R}$ are continuous at a point $c \in A$. Then,
    \begin{enumerate}
        \item $kf(x)$ is continuous at $c$ for all $k \in \mathbb{R}$;
        \item $f(x) + g(x)$ is continuous at $c$;
        \item $f(x)g(x)$ is continuous at $c$; and
        \item $f(x)/g(x)$ is continuous at $c$, provided the quotient is defined.
        
        Proof on page 124.
    \end{enumerate}
\end{theorem}

\begin{theorem}[Composition of Continuous Functions - 4.3.9]
    Given $f:A\rightarrow \mathbb{R}$ and $g:B\rightarrow \mathbb{R}$, assume that the range $f(A) = \{f(x):x\in A\}$ is contained in the domain $B$ so that the composition $g \circ f(x) = g(f(x))$ is defined on $A$.

    If $f$ is continuous at $c \in A$, and if $g$ is continuous at $f(c) \in B$, then $g \circ f$ is continuous at $c$.
\end{theorem}

\end{document}