\documentclass{article}
\usepackage[utf8]{inputenc}
\usepackage{amsmath}
\usepackage{amssymb}
\usepackage{amsthm}
\usepackage[shortlabels]{enumitem}
\usepackage[margin=1in]{geometry}
\setlist*[enumerate]{label=(\alph*)}

\title{Math 320 Exam 2 Notes}
\author{Christopher Kapic}
\date{October 2020}

\begin{document}
\maketitle

\section*{Geometric Series}
\begin{enumerate}
    \item Ratio test: In a geometric series $\sum _{n=0}^\infty a_n$, we can find the common ratio $r = \frac{a_{n+1}}{a_n}$. If $r < 1$ then the series converges. If, however, $r \geq$ 1 then the series diverges.
    \item Let $(a_n), (b_n)$ be sequences, and $\lim a_n = a, \lim b_n = b$. Then:
        \begin{enumerate}[(i)]
            \item $\lim _{n\rightarrow\infty} c a_n = ca$
            \item $\lim _{n\rightarrow\infty} a_n + b_n = a+b$
            \item $\lim _{n\rightarrow\infty} a_nb_n = ab$
            \item $\lim _{n\rightarrow\infty} \frac{a_n}{b_n}=\frac{a}{b}$ (as long as $b, \lim b_n \neq 0$)
        \end{enumerate}
\end{enumerate}

\section*{Squeeze theorem}
If we are trying to find $\lim f(x)$ and we know $g(x)\leq f(x) \leq h(x)$ and $\lim g(x)=\lim h(x)$, then $\lim f(x)=\lim g(x)=\lim h(x).$


\section*{Binomial Expansion}
$(a+b)^n=\sum _{k=0}^n\frac{n!}{k!(n-k)!}a^{n-k}b^k$.

\section*{Sequences}
\begin{enumerate}
    \item A sequence is a function whose domain is $\mathbb{N}$.
    \item Convergence: A sequence $(a_n)$ converges to $a$ if for all $\epsilon > 0$ there exists $N \in \mathbb{N}$ such that for all $n > N, |a_n - a| < \epsilon$.
    \item This one is kind of obvious, but when there is a limit of a sequence, it must be unique (there cannot be two limits of a sequence).
    \item Divergence: A sequence that does not converge diverges.
    \item Boundedness: A sequence $x_n$ is bounded if there exists an $M > 0$ such that $|x_n|\leq M$ for all $n \in \mathbb{N}$.
    \item Every convergent sequence is bounded.
    \item Algebraic limit theorem: see Geometric Series (b). Proof on page 50 of the book.
\end{enumerate}

\section*{Limits and Order}
\begin{enumerate}
    \item Order Limit theorem | assume $\lim a_n = a, \lim b_n = b$
        \begin{enumerate}[(i)]
            \item If $a_n \geq 0 \forall n \in \mathbb{N}, a \geq 0.$
            \item If $a_n < b_n \forall n \in \mathbb{N}, a < b$.
            \item (For $c$) If $c \leq b_n$ for all $n \in \mathbb{N}$, then $c \leq b$. Likewise, if $a_n \leq c$ for all $n \in \mathbb{N}$, then $a \leq c.$
        \end{enumerate}
        Proof on page 53.
    \item The algebraic limit theorem ensures that $(b_n - a_n)$ converges to $b-a$.
\end{enumerate}

\section*{Monotone convergence}
See page 56.

\end{document}
