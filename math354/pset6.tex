\documentclass{article}
\usepackage[utf8]{inputenc}
\usepackage{amsmath}
\usepackage{amssymb}
\usepackage{amsthm}
\newtheorem{definition}{Definition}
\newtheorem{theorem}{Theorem}
\newtheorem{hint}{Hint}
\usepackage[shortlabels]{enumitem}
\usepackage[margin=1in]{geometry}
\setlist*[enumerate]{label=(\alph*)}

\title{Math 354 Homework 6 Notes}
\author{Christopher Kapic}
\date{November 2020}

\begin{document}

\maketitle

\section*{10.1 Chaos}

\begin{definition}[dense]
    Suppose $X$ is a set and $Y$ is a subset of $X$. We say that $Y$ is \textit{dense} in $X$ if, for any point $x \in X$, there is a point $y$ in the subset $Y$ arbitrarily close to $x$.
\end{definition}

\begin{definition}[transitivs]
    A dynamical system is \textit{transitive} if for any pair of points $x$ and $y$ and any $\epsilon > 0$ there is a third point $z$ within $\epsilon$ of $x$ whose orbit comes within $\epsilon$ of $y$.
\end{definition}

\begin{definition}[sensitive]
    A dynamical system $F$ \textit{depends sensitively on initial conditions} if there is a $\beta > 0$ such that, for any $x$ and any $\epsilon > 0$, there is a $y$ within $\epsilon$ of $x$ and a $k$ such that the distance between $F^k(x)$ and $F^k(y)$ is at least $\beta$.
\end{definition}

\begin{definition}[chaotic]
    A dynamical system $F$ is \textit{chaotic} if:
    \begin{enumerate}[(i)]
        \item Periodic points for $F$ are dense.
        \item $F$ is transitive.
        \item $F$ depends sensitively on initial conditions.
    \end{enumerate}
\end{definition}

\begin{theorem}
    The shift map $\sigma :\Sigma \rightarrow \Sigma$ is a chaotic dynamical system.
\end{theorem}

\begin{theorem}[The Density Proposition]
    Suppose $F:X\rightarrow Y$ is a continuous map that is onto and suppose also that $D \subset X$ is a dense subset. Then $F(D)$ is dense in $Y$.\\\\
    Proof on page 126.
\end{theorem}

\begin{theorem}
    Suppose $c < -(5+2\sqrt{5})/4$. Then the quadratic map $Q_c(x) =x^2+c$ is chaotic on the set $\Lambda$.\\\\
    Proof on page 126.
\end{theorem}

\section*{10.2 Other Chaotic Systems}

\begin{theorem}
    The function $Q_{-2}(x)=x^2-2$ is chaotic on $[-2,2]$.
\end{theorem}

\begin{definition}[semiconjugacy]
    Suppose $F:X\rightarrow X$ and $G:Y\rightarrow Y$ are two dynamical systems. A mapping $h:X\rightarrow Y$ is called a \textit{semiconjugacy} if $h$ is continuous, onto, at most $n$-to-one, and satisfies $h \circ F = G\circ h$.
\end{definition}

\begin{theorem}
    The doubling map $D$ is chaotic on the unit circle.\\\\
    Proof on page 131.
\end{theorem}

\section*{10.3 Manifestations of Chaos}

\begin{definition}[superstable]
    Suppose $x_0$ is a critical point for $F$, that is, $F'(x_0) = 0$. If $x_0$ is also a periodic point of $F$ with period $n$, then the orbit of $x_0$ is called \textit{superstable}. The reason for this terminology is taht $(F^n)'(x_0)=0$.
\end{definition}


\end{document}
